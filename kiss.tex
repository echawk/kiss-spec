\documentclass{article}
\title{The KISS Package Manager \& System}
\author{Ethan Hawk et al.}

% TODO: come up w/ sane bold, italics, etc scheme.

\begin{document}
\maketitle
\tableofcontents

\newpage
This document seeks to define the qualities that constitute the
KISS package manager as well the KISS system.

It is also a reference for anyone who is looking to implement
either part of the KISS system.

\section{History}

KISS, both the package manager and system trace their origin back
to 2019, and were created by Dylan Araps.

Since their creation in 2019, there has been rather little change
in both the format and the structure of the system. The biggest
change that impacts this document was the migration from using
sha256 checkums to using b3sums in September of 2022.

There was also a brief period, from July of 2021 to March of 2022
where the sources file supported a DSL that allowed the maintainer
to have the version of the package be automatically resolved just
by updating the version file. This feature was added by Dylan,
and removed by the community after his second hiatus. The reason for
its removal was that VERSION markers made certain tasks with
the system somewhat trickier, as well as complicating the packge format.
For historical reasons, this format will be documented in this document,
however this behavior has long since been deprecated.

While this section is not strictly necessary for the definition of
either the package manager or the system, I think this section ought
to exist, so that way future readers and implementers can
gleam some insight into the why and how KISS became what it is today.

\section{KISS System}

This section is still somewhat contested, as there has yet to
be any discussion in the wider KISS community as to what
\textit{exactly} should be available on a system for it
to be considered a KISS System.

Generally speaking, the following is expected:

\begin{itemize}
\item POSIX core utilites
\item git
\item curl
\item An implementation of SSL
\item gzip, bzip2, xz, \& zstd
\end{itemize}

It is conceivable to run kiss, the package manager, on any system
that meets these requirements.

It is important to note that this list is not yet comprehensive, and
there is also an expectation that whatever system you are running
the package manager on will also have access to a C compiler and
additional POSIX amenities.

In the future there will ideally be some effort put into making
a package, which when installed, can verify that the system is
compliant.

\section{KISS Package Manager}

For this section, unless otherwise clarified, KISS refers to the
package manager.

KISS the package manager is a source-based package manager
not too dissimilar from portage from Gentoo, or the ports
system from the *BSDs. Unlike those systems however,
KISS has no system wide configuration, instead relying
on the existence of environment variables to change it's behavior.

\subsection{System Requirements}

It is assumed that KISS will be running on a system that meets
the requirements of the to be mentioned \textit{KISS System}.

\subsection{Environment Variables}

KISS is entirely configured through the existence of these environment
variables. What follows is a table of each environment variable along
with a brief description of what the said variable does.

\pagebreak
\begin{center}
\begin{table}[]
\begin{tabular}{|l|l|}
  \hline
  Variable & Description \\ \hline
  KISS\_CHK      & Utility to use when checksumming sources \\
  KISS\_CHOICE   & Enables or Disables the alternatives system \\
  KISS\_COLOR    & Enable or Disable colors \\
  KISS\_COMPRESS & Compression method to use for built tarballs \\
  KISS\_DEBUG    & Keep temporary directories around (debugging purposes) \\
  KISS\_ELF      & Which readelf command to use \\
  KISS\_FORCE    & Force the installation/removal of packages \\
  KISS\_GET      & Utility to use when downloading sources \\
  KISS\_HOOK     & Colon separated list of absolute paths to executable files \\
  KISS\_KEEPLOG  & Keep build logs around for successful builds and not just \\
  KISS\_PATH     & List of repositories. Works exactly like \$PATH \\
  KISS\_PROMPT   & Enables or Disables prompts from the package manager \\
  KISS\_ROOT     & Where installed packages will go \\
  KISS\_STRIP    & Enable or Disable package stripping globally \\
  KISS\_SU       & The sudo-like utility to use for privilege escalation \\
  KISS\_TMPDIR   & Temporary directory for builds \\
  \hline
\end{tabular}
\end{table}
\end{center}

The following table denotes the valid values for each of the
variables, where applicable.

\begin{center}
\begin{table}[]
\begin{tabular}{|l|l|l|}
  \hline
  Variable       & Default Value & Valid Values \\ \hline
  KISS\_CHK      & openssl       & openssl, sha256sum, sha256, shasum, digest \\
  KISS\_CHOICE   & 1             & 0, 1                                       \\
  KISS\_COLOR    & 1             & 0, 1                                       \\
  KISS\_COMPRESS & gz            & gz, bz2, lzma, lz, xz, zst                 \\
  KISS\_DEBUG    & 0             & 0, 1                                       \\
  KISS\_ELF      & readelf       & readelf, readelf-*, ldd                    \\
  KISS\_FORCE    & 0             & 0, 1                                       \\
  KISS\_GET      & curl          & aria2c, axel, curl, wget, wget2            \\
  KISS\_HOOK     & ""            & (anything)                                 \\
  KISS\_KEEPLOG  & 0             & 0, 1                                       \\
  KISS\_PATH     & ""            & (anything)                                 \\
  KISS\_PROMPT   & 1             & 0, 1                                       \\
  KISS\_ROOT     & "/"           & (anything)                                 \\
  KISS\_STRIP    & 1             & 0, 1                                       \\
  KISS\_SU       & ""            & ssu, sudo, doas, su                        \\
  KISS\_TMPDIR   & ""            & (anything)                                 \\
  \hline
\end{tabular}
\end{table}
\end{center}

\pagebreak
\subsection{Search Resolution}

The environment variable \textbf{KISS\_PATH} is responsible for determining
where a particular package will be sourced from. Directories earlier
in \textbf{KISS\_PATH} will be searched first for the package, in a similar
way to how the system PATH is searched for an executable.

\subsection{Package Format}

This section of the document will document and specify the different files
that go into defining a KISS package port.

\subsubsection{version}

The \texttt{version} file is a plain text file (or a symbolic link to such a
file) whose first line contains the version and release of the port.

For example, a file containing the following:

\begin{verbatim}
1.0.4 3
\end{verbatim}

Would be read to mean that the package is on version 1.0.4, with 3 signifying
the third release.

Generally, releases only ought to occur if a dependency of the package in
question breaks the installed package, thus necessitating a rebuild. Or
if there is additional functionality enabled/disabled.

\subsubsection{sources}

The \texttt{sources} file is a plain text file (or a symbolic link to such a
file) whose text contains a \textit{source} that the package needs for it to
build.

The sources file will be divided on a line by line basis. Each line
indicates a unique source. Each source can optionally specify a directory
that it will then be extracted or copied into during the build of the package.

Package source types are split into two general categories: remote \& local.

Local sources can either be an absolute path on the file system, such as
\texttt{/path/to/package/source.file\_extension} or can be a relative path
such as \texttt{files/source-file}.

Remote sources can either be URL to a remote file, or can be the link
to a git repo, provided that the URL is appropriately prefixed with the
string \texttt{git+}.

The \texttt{\@} and \texttt{\#} are used to separate the url of the repository
from either the commit or branch that should be checked out. Otherwise, if
that part of the source is blank, the HEAD of the repository is fetched instead.

It should be noted that there is no difference between \texttt{\@} and \texttt{\#}
when it comes to the internals of the package manager, it is merely a stylistic
choice.

Here is an example sources file:

\begin{verbatim}
https://github.com/godotengine/godot/archive/refs/tags/4.2.0-stable.tar.gz
git+https://github.com/SCons/scons scons/
patches/gcc.patch
\end{verbatim}

\subsubsection{depends}

The \texttt{depends} file contains a list of all of the dependencies for
the port, separated by newlines. Dependencies in this package can be marked
as a ``make'' dependency, indicating that the package is only required by
the port at build time, and can be removed after the package has been built.
Otherwise, dependencies are assumed to be runtime dependencies, which cannot
be removed when the package is installed.

It is important to note that there is no way to indicate optional dependencies
in this scheme.

Here is an example depends file:
\begin{verbatim}
python
meson make
util-linux
\end{verbatim}

\subsubsection{checksums}

The \texttt{checksums} file contains the b3sums for each source in the
sources file.

\subsubsection{pre-remove}

\subsubsection{post-install}

\subsubsection{build}

The \texttt{build} file is responsible for building the port.
The only requirement of the build file is that it be marked as an executable.

The build file is given to arguments when executed:
\begin{enumerate}
\item A temporary DESTDIR to install the package contents to.
\item The version of the package being built.
\end{enumerate}

\subsection{Hooks}

\begin{center}
\begin{table}[]
\begin{tabular}{|l|l|l|l|l|}
  \hline
  hook          & arg1   & arg2     & arg3               & arg4           \\
  \hline
  build-fail    & Type   & Package  & Build directory    &                \\
  post-build    & Type   & Package  & DESTDIR            &                \\
  post-install  & Type   & Package  & Installed database &                \\
  post-package  & Type   & Package  & Tarball            &                \\
  post-source   & Type   & Package  & Verbatim source    & Resolved source\\
  post-update   & Type   & [7]      &                    &                \\
  pre-build     & Type   & Package  & Build directory    &                \\
  pre-extract   & Type   & Package  & DESTDIR            &                \\
  pre-install   & Type   & Package  & Extracted package  &                \\
  pre-remove    & Type   & Package  & Installed database &                \\
  pre-source    & Type   & Package  & Verbatim source    & Resolved source\\
  pre-update    & Type   & [7] [8]  &                    &                \\
  queue-status  & Type   & Package  & Number in queue    & Total in queue \\
  \hline
\end{tabular}
\end{table}
\end{center}

%The -update hooks start in the current repository. In other words, you can
%operate on the repository directly or grab the value from '\$PWD'.

%The second argument of pre-update is '0' if the current user owns the
%repository and '1' if they do not. In the latter case, privilege
%escalation is required to preserve ownership.


\end{document}
